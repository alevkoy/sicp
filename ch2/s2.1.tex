\documentclass{article}
\author{Abraham Levkoy}
\title{SICP exercises, section 2.1}

\usepackage{mystyle}

\begin{document}
\maketitle

\section{Exercise 2.1}
\begin{quote}
    Define a better version of \texttt{make-rat} that handles both positive and
    negative arguments. \texttt{Make-rat} should normalize the sign so that if
    the rational number is positive, both the numerator and denominator are
    positive, and if the rational number is negative, only the numerator is
    negative.
\end{quote}

\lstinputlisting[firstline=22,lastline=35]{ch2/ex2.1.scm}

\section{Exercise 2.2}
\begin{quote}
    Consider the problem of representing line segments in a plane. Each segment
    is represented as a pair of points: a starting point and an ending point.
    Define a constructor \texttt{make-segment} and selectors
    \texttt{start-segment} and \texttt{end-segment} that define the
    representation of segments in terms of points.  Furthermore, a point can be
    represented as a pair of numbers: the $x$ coordinate and the $y$
    coordinate. Accordingly, specify a constructor \texttt{make-point} and
    selectors \texttt{x-point} and \texttt{y-point} that define this
    representation. Finally, using your selectors and constructors, define a
    procedure \texttt{midpoint-segment} that takes a line segment as argument
    and returns its midpoint (the point whose coordinates are the average of
    the coordinates of the endpoints). To try your procedures, you’ll need a
    way to print points:
    \begin{lstlisting}
(define (print-point p)
  (newline)
  (display "(")
  (display (x-point p))
  (display ",")
  (display (y-point p))
  (display ")"))
    \end{lstlisting}
\end{quote}

\lstinputlisting[firstline=27,lastline=53]{ch2/ex2.2.scm}

\section{Exercise 2.3}
\begin{quote}
    Implement a representation for rectangles in a plane. (Hint: You may want
    to make use of Exercise 2.2.) In terms of your constructors and selectors,
    create procedures that compute the perimeter and the area of a given
    rectangle. Now implement a different representation for rectangles. Can you
    design your system with suitable abstraction barriers, so that the same
    perimeter and area procedures will work using either representation?
\end{quote}

The first implementation represents a rectangle as 3 points. The fourth point
could be derived if it were needed. The \texttt{area} and \texttt{perimeter}
functions are implemented in terms of helpers \texttt{height-rectangle} and
\texttt{width-rectangle}, which are in turn implemented in terms of the point
selectors for this implementation.
% Still working in ex2.2.scm to avoid copying solution of exercise 2.2
\lstinputlisting[firstline=64,lastline=100]{ch2/ex2.2.scm}

The second implementation represents a rectangle as 2 segments, which are
assumed to share a point. This requires new selectors and new implementations of
\texttt{width-rectangle} and \texttt{height-rectangle}, but \texttt{area} and
\texttt{perimeter} don't need to change.
\lstinputlisting[firstline=120,lastline=133]{ch2/ex2.2.scm}

\section{Exercise 2.4}
\begin{quote}
    Here is an alternative procedural representation of pairs. For this
    representation, verify that \texttt{(car (cons x y))} yields \texttt{x} for
    any objects \texttt{x} and \texttt{y}.
    \begin{lstlisting}
(define (cons x y)
  (lambda (m) (m x y)))

(define (car z)
  (z (lambda (p q) p)))
    \end{lstlisting}
    What is the corresponding definition of \texttt{cdr}? (Hint: To verify that
    this works, make use of the substitution model of 1.1.5.)
\end{quote}

\begin{lstlisting}
(car (cons x y))
(car (lambda (m) (m x y)))
((lambda (m) (m x y)) (lambda (p q) p))
((lambda (p q) p) x y)
x
\end{lstlisting}
\texttt{car} produces the expected result. The corresponding definition of
\texttt{cdr} is
\lstinputlisting[firstline=10,lastline=11]{ch2/ex2.4.scm}

\section{Exercise 2.5}
\begin{quote}
    Show that we can represent pairs of nonnegative integers using only numbers
    and arithmetic operations if we represent the pair $a$ and $b$ as the
    integer that is the product $2^a3^b$. Give the corresponding definitions of
    the procedures \texttt{cons}, \texttt{car}, and \texttt{cdr}.
\end{quote}

2 and 3 are coprime, so $a$ and $b$ can be found by finding the umber of times
that 2 and 3, respectively, divide evenly into the whole number.
\lstinputlisting[firstline=3,lastline=16]{ch2/ex2.5.scm}

\section{Exercise 2.6}
\begin{quote}
    In case representing pairs as procedures wasn't mind-boggling enough,
    consider that, in a language that can manipulate procedures, we can get by
    without numbers (at least insofar as nonnegative integers are concerned) by
    implementing 0 and the operation of adding 1 as
    \begin{lstlisting}
(define zero (lambda (f) (lambda (x) x)))

(define (add-1 n)
  (lambda (f) (lambda (x) (f ((n f) x)))))
    \end{lstlisting}
    This representation is known as Church numerals, after its inventor, Alonzo
    Church, the logician who invented the $\lambda$-calculus.

    Define \texttt{one} and \texttt{two} directly (not in terms of
    \texttt{zero} and \texttt{add-1}). (Hint: Use substitution to evaluate
    \texttt{(add-1 zero)}). Give a direct definition of the addition procedure
    \texttt{+} (not in terms of repeated application of \texttt{add-1}).
\end{quote}

A Church numeral represents a number as a procedure that applies f to its input
as many times as the number.
\begin{lstlisting}
(add-1 zero)
(lambda (f) (lambda (x) (f ((zero f) x))))
(lambda (f) (lambda (x) (f (((lambda (f) (lambda (x) x)) f) x))))
(lambda (f) (lambda (x) (f (((lambda (f) f)) x))))
(lambda (f) (lambda (x) (f x)))
\end{lstlisting}
\texttt{one} is the procedure of \texttt{f} that returns a procedure of
\texttt{x} that applies \texttt{f} to \texttt{x} once. Similarly, \texttt{two}
is a procedure that applies \texttt{f} twice.
\lstinputlisting[firstline=15,lastline=16]{ch2/ex2.6.scm}

Addition is defined below. I did not redefine \texttt{+} to avoid breaking any
built-in procedures or making life generally uncomfortable for myself.
\lstinputlisting[firstline=19,lastline=26]{ch2/ex2.6.scm}

\section{Exercise 2.7}
\begin{quote}
    Alyssa's program is incomplete because she has not specified the
    implementation of the interval abstraction. Here is a definition of the
    interval constructor:
    \begin{lstlisting}
(define (make-interval a b) (cons a b))
    \end{lstlisting}
    Define selectors \texttt{upper-bound} and \texttt{lower-bound} to complete
    the implementation.
\end{quote}

\lstinputlisting[firstline=30,lastline=34]{ch2/ex2.7.scm}
Ordering the lower bound before the upper bound is concistent with the usage of
\texttt{lower-bound} and \texttt{upper-bound} in \texttt{add-interval}, as well
as the other provided interval methods.

\section{Exercise 2.8}
\begin{quote}
    Using reasoning analogous to Alyssa's, describe how the difference of two
    intervals may be computed. Define a corresponding subtraction procedure,
    called \texttt{sub-interval}.
\end{quote}

% Still working in ex2.7.scm to avoid copy-pasting
\lstinputlisting[firstline=60,lastline=64]{ch2/ex2.7.scm}

\section{Exercise 2.9}
\begin{quote}
    The width of an interval is half of the difference between its upper and
    lower bounds. The width is a measure of the uncertainty of the number
    specified by the interval. For some arithmetic operations the width of the
    result of combining two intervals is a function only of the widths of the
    argument intervals, whereas for others the width of the combination is not
    a function of the widths of the argument intervals. Show that the width of
    the sum (or difference) of two intervals is a function only of the widths
    of the intervals being added (or subtracted). Give examples to show that
    this is not true for multiplication or division.
\end{quote}

\subsection{Addition}
\begin{equation*}
    % This doesn't produce exactly what I'd like formatting-wise, but it's close
    % enough.
    \begin{split}
        \textrm{Let }i_a = [l_a, h_a], i_b = [l_b, h_b] \\
        \Rightarrow width_a = h_a - l_a, width_b = h_b - l_b \\
        i_{a+b} = [l_a + l_b, h_a + h_b] \\
        \Rightarrow width_{a+b} = (h_a + h_b) - (l_a + l_b) \\
                                = (h_a - l_a) + (h_b - l_b) \\
                                = width_a + width_b \\
    \end{split}
\end{equation*}
$width_{a+b}$ is a function only of $width_a$ and $width_b$.

\subsection{Subtraction}
\begin{equation*}
    \begin{split}
        \textrm{Let }i_a = [l_a, h_a], i_b = [l_b, h_b] \\
        \Rightarrow width_a = h_a - l_a, width_b = h_b - l_b \\
        i_{a-b} = [l_a - l_b, h_a - h_b] \\
        \Rightarrow width_{a+b} = (h_a - h_b) - (l_a - l_b) \\
                                = (h_a - l_a) - (h_b - l_b) \\
                                = width_a - width_b \\
    \end{split}
\end{equation*}
$width_{a-b}$ is a function only of $width_a$ and $width_b$.

\subsection{Multiplication}
\begin{equation*}
    \begin{split}
        \textrm{Let }i_a = [2, 3], i_b = [3, 5], i_c = [4, 6] \\
        \Rightarrow width_b = width_c = 2 \\
        i_{a \times b} = [6, 15], i_{a \times c} = [8, 18] \\
        \Rightarrow width_{a \times b} = 9, width_{a \times c} = 10 \\
    \end{split}
\end{equation*}
$width_{a \times b}$ is not just a function of $width_a$ and $width_b$.

\subsection{Division}
\begin{equation*}
    \begin{split}
        \textrm{Let }i_a = [2, 3], i_b = [3, 5], i_c = [4, 6] \\
        \Rightarrow width_b = width_c = 2 \\
        i_{a \div b} = [2/5, 1], i_{a \div c} = [1/3, 3/4] \\
        \Rightarrow width_{a \div b} = 3/5, width_{a \div c} = 5/12 \\
    \end{split}
\end{equation*}
$width_{a \div b}$ is not just a function of $width_a$ and $width_b$.

\section{Exercise 2.10}
\begin{quote}
    Ben Bitdiddle, an expert systems programmer, looks over Alyssa's shoulder
    and comments that it is not clear what it means to divide by an interval
    that spans zero. Modify Alyssa's code to check for this condition and to
    signal an error if it occurs.
\end{quote}

\lstinputlisting[firstline=73,lastline=85]{ch2/ex2.7.scm}

\section{Exercise 2.11}
\begin{quote}
    In passing, Ben also cryptically comments: “By testing the signs of the
    endpoints of the intervals, it is possible to break \texttt{mul-interval}
    into nine cases, only one of which requires more than two multiplications.”
    Rewrite this procedure using Ben’s suggestion.
\end{quote}

\lstinputlisting[firstline=94,lastline=143]{ch2/ex2.7.scm}

\section{Exericise 2.12}
\begin{quote}
    Define a constructor \texttt{make-center-percent} that takes a center and a
    percentage tolerance and produces the desired interval. You must also
    define a selector \texttt{percent} that produces the percentage tolerance
    for a given interval. The \texttt{center} selector is the same as the one
    shown above.
\end{quote}

\lstinputlisting[firstline=204,lastline=217]{ch2/ex2.7.scm}

\section{Exercise 2.13}
\begin{quote}
    Show that under the assumption of small percentage tolerances there is a
    simple formula for the approximate percentage tolerance of the product of
    two intervals in terms of the tolerances of the factors. You may simplify
    the problem by assuming that all numbers are positive.
\end{quote}

\begin{proof}
    Let $i_1 = [a,b], i_2 = [c,d]$. Since all bounds are positive, we may use
    \begin{equation*}
        i_3 = i_1 * i_2 = [ac,bd]
    \end{equation*}
    for multiplication.

    Let $p_1$, $p_2$, and $p_3$ be the percentage tolerances of $i_1$, $i_2$,
    and $i_3$, respectively.
    \begin{IEEEeqnarray*}{rrClCl}
        \Rightarrow & p_1 &=& \frac{(b-a)/2}{(b+a)/2} & = & \frac{b-a}{b+a} \\
                    & p_2 &=&&& \frac{d-c}{d+c} \\
                    & p_3 &=&&& \frac{bd-ac}{bd+ac} \\
    \end{IEEEeqnarray*}
    \begin{IEEEeqnarray*}{rCl}
        p_1 + p_2 &=& \frac{b-a}{b+a} + \frac{d-c}{d+c} \\
                  &=& \frac{(b-a)(d+c) + (d-c)(b+a)}{(b+a)(d+c)} \\
                  &=& \frac{bd+bc-ad-ac+bd+ad-bc-ac}{bd+bc+ad+ac} \\
                  &=& \frac{2(bd-ac)}{bd+ac+bc+ad} \\
    \end{IEEEeqnarray*}

    It is given that the percentage tolerances are small. Among other
    things, this means that $c \approx d$.
    \begin{IEEEeqnarray*}{rCl}
        \Rightarrow p_1 + p_2 &\approx& \frac{2(bd-ac)}{bd+ac+bd+ac} \\
        &=& \frac{2(bd-ac)}{2(bd+ac)} \\
        &=& \frac{bd-ac}{bd+ac} \\
        &=& p_3 \\
        \Rightarrow p_3 &\approx& p_1 + p_2 \\
    \end{IEEEeqnarray*}
\end{proof}

\end{document}
