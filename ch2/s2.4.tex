\documentclass{article}
\author{Abraham Levkoy}
\title{SICP exercises, section 2.4}

\usepackage{mystyle}

\begin{document}
\maketitle

\section{Exercise 2.73}
\begin{quote}
    2.3.2 described a program that performs symbolic differentiation:

    \begin{lstlisting}
(define (deriv exp var)
  (cond ((number? exp) 0)
        ((variable? exp)
         (if (same-variable? exp var) 1 0))
        ((sum? exp)
         (make-sum (deriv (addend exp) var)
                   (deriv (augend exp) var)))
        ((product? exp)
         (make-sum
           (make-product
            (multiplier exp)
            (deriv (multiplicand exp) var))
           (make-product
            (deriv (multiplier exp) var)
            (multiplicand exp))))
        ⟨more rules can be added here⟩
        (else (error "unknown expression type:
                      DERIV" exp))))
    \end{lstlisting}

    We can regard this program as performing a dispatch on the type of the
    expression to be differentiated. In this situation the ``type tag'' of the
    datum is the algebraic operator symbol (such as \texttt{+}) and the
    operation being performed is \texttt{deriv}. We can transform this program
    into data-directed style by rewriting the basic derivative procedure as

    \begin{lstlisting}
(define (deriv exp var)
   (cond ((number? exp) 0)
         ((variable? exp)
           (if (same-variable? exp var)
               1
               0))
         (else ((get 'deriv (operator exp))
                (operands exp)
                var))))

(define (operator exp) (car exp))
(define (operands exp) (cdr exp))
    \end{lstlisting}

    \begin{enumerate}
        \item Explain what was done above. Why can’t we assimilate the
            predicates \texttt{number?} and \texttt{variable?} into the
            data-directed dispatch?
        \item Write the procedures for derivatives of sums and products, and
            the auxiliary code required to install them in the table used by
            the program above.
        \item Choose any additional differentiation rule that you like, such as
            the one for exponents (Exercise 2.56), and install it in this
            data-directed system.
        \item In this simple algebraic manipulator the type of an expression is
            the algebraic operator that binds it together. Suppose, however, we
            indexed the procedures in the opposite way, so that the dispatch
            line in \texttt{deriv} looked like

            \begin{lstlisting}
((get (operator exp) 'deriv)
 (operands exp) var)
            \end{lstlisting}

            What corresponding changes to the derivative system are required?
    \end{enumerate}
\end{quote}

\begin{enumerate}
    \item The data-directed version above dispatches on the operator of the
        expression. Expressions for which \texttt{number?} or \texttt{variable?}
        return true don't have operators.
    \item \texttt{deriv} package for sums and products:
        \lstinputlisting[firstline=66,lastline=109]{ch2/ex2.73.scm}
    \item \texttt{deriv} package for exponentiations:
        \lstinputlisting[firstline=129,lastline=154]{ch2/ex2.73.scm}
    \item If the order of the type and operation in the \texttt{get} calls were
        changed, the only corresponding change necessary would be a similar
        transposition in the \texttt{put} calls in the implementations of
        \texttt{deriv} for each type of expression.
\end{enumerate}

\section{Exerise 2.74}
\begin{quote}
    Insatiable Enterprises, Inc., is a highly decentralized conglomerate
    company consisting of a large number of independent divisions located all
    over the world. The company’s computer facilities have just been
    interconnected by means of a clever network-interfacing scheme that makes
    the entire network appear to any user to be a single computer. Insatiable's
    president, in her first attempt to exploit the ability of the network to
    extract administrative information from division files, is dismayed to
    discover that, although all the division files have been implemented as
    data structures in Scheme, the particular data structure used varies from
    division to division. A meeting of division managers is hastily called to
    search for a strategy to integrate the files that will satisfy
    headquarters' needs while preserving the existing autonomy of the
    divisions.

    Show how such a strategy can be implemented with data-directed programming.
    As an example, suppose that each division's personnel records consist of a
    single file, which contains a set of records keyed on employees' names. The
    structure of the set varies from division to division. Furthermore, each
    employee’s record is itself a set (structured differently from division to
    division) that contains information keyed under identifiers such as address
    and salary. In particular:

    \begin{enumerate}
        \item Implement for headquarters a \texttt{get-record} procedure that
            retrieves a specified employee's record from a specified personnel
            file. The procedure should be applicable to any division's file.
            Explain how the individual divisions' files should be structured.
            In particular, what type information must be supplied?
        \item Implement for headquarters a \texttt{get-salary} procedure that
            returns the salary information from a given employee's record from
            any division's personnel file. How should the record be structured
            in order to make this operation work?
        \item Implement for headquarters a \texttt{find-employee-record}
            procedure. This should search all the divisions' files for the
            record of a given employee and return the record. Assume that this
            procedure takes as arguments an employee's name and a list of all
            the divisions' files.
        \item When Insatiable takes over a new company, what changes must be
            made in order to incorporate the new personnel information into the
            central system?
    \end{enumerate}
\end{quote}

\begin{enumerate}
    \item \texttt{get-record} determines the division that created \texttt{file}
        and applies the appropriate procedure. In order for \texttt{get-record}
        to work, each file must be tagged with the division it came from.

        \lstinputlisting[firstline=25,lastline=80]{ch2/ex2.74.scm}

        In this example, Division A stores records in an unordered list with
        each record a list of name and salary, in that order. Division B stores
        records as a binary search tree, ordered by employee name, with each
        record a pair of employee name and salary, in that order.
    \item Records do not need to have any particular structure, since
        \texttt{get-salary} will dispatch on the type of record for the file
        from which the record was obtained. An alternative approach would be to
        have \texttt{get-record} tag each record it produces with the division
        from which it was obtained. This is redundant for the limited use cases
        here, but it could be useful if records from different divisions were
        mixed together and operated upon in bulk.

        \lstinputlisting[firstline=132,lastline=157]{ch2/ex2.74.scm}
    \item This is a straightforward application of the previously defined
        \texttt{get-record}.

        \lstinputlisting[firstline=180,lastline=186]{ch2/ex2.74.scm}
    \item When Insatiable acquires a new company, \texttt{get-record} and
        \texttt{get-salary} procedures will need to be written for that
        company's personnel file. If, as seems likely, that file does not
        already exist in the form of a Scheme data structure, it will need to
        be converted into one, and that structure will need to be tagged with
        the name of the new division to enable dispatch.

        This architecture works for records with a small number of fields that
        only need to be processed in limited ways. I think a more flexible
        architecture would be to develop a record format that will support the
        fields of all divisions. Then, each division defines a package to
        translate their own records into the unified format. Any subsequent
        operations could be performed on collections of translated records from
        many divisions.
\end{enumerate}

\section{Exercise 2.75}
\begin{quote}
    Implement the constructor \texttt{make-from-mag-ang} in message-passing
    style. This procedure should be analogous to the
    \texttt{make-from-real-imag} procedure given above.
\end{quote}

\lstinputlisting[firstline=18,lastline=26]{ch2/ex2.75.scm}

\section{Exercise 2.76}
\begin{quote}
    As a large system with generic operations evolves, new types of data
    objects or new operations may be needed. For each of the three strategies
    -- generic operations with explicit dispatch, data-directed style, and
    message-passing-style -- describe the changes that must be made to a system
    in order to add new types or new operations. Which organization would be
    most appropriate for a system in which new types must often be added? Which
    would be most appropriate for a system in which new operations must often
    be added?
\end{quote}

Steps to add a new type or operation:
\begin{itemize}
    \item For explicit dispatch, a new type requires new constructors and
        accessors, and the implementation of each higher-level operation must
        be modified to support the new type. A new low-level operation requires
        an implementation that is aware of every type in the system.
    \item For data-directed style, a new type requires an implementation of
        every data-directed accessor and a new constructor. A new low-level
        operation requires an implementation for every type.
    \item For message-passing style, a new type requires a new implementation of
        every low-level operation. A new low-level operation requires an
        implementation in every type.
\end{itemize}

If new types must often be added, data-directed or message-passing styles would
be appropriate. In each case, the required modifications can be made in one
place without disturbing the code in the existing system.

If new low-level operations must be added often, explicit dispatch would be more
appropriate, also because it isolates the changes from the existing code.

Assuming appropriate layering, new high-level operations can be added equally
easily in each style.

\end{document}
