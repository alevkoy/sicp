\documentclass{article}
\author{Abraham Levkoy}
\title{SICP exercises, section 2.4}

\usepackage{mystyle}

\begin{document}
\maketitle

\section{Exercise 2.77}
\begin{quote}
    Louis Reasoner tries to evaluate the expression \texttt{(magnitude z)} where
    \texttt{z} is the object shown in Figure 2.24. To his surprise, instead of
    the answer 5 he gets an error message from \texttt{apply-generic}, saying
    there is no method for the operation \texttt{magnitude} on the types
    \texttt{(complex)}. He shows this interaction to Alyssa P. Hacker, who says
    ``The problem is that the complex-number selectors were never defined for
    \texttt{complex} numbers, just for \texttt{polar} and \texttt{rectangular}
    numbers. All you have to do to make this work is add the following to the
    \texttt{complex} package:''
    \begin{lstlisting}
(put 'real-part '(complex) real-part)
(put 'imag-part '(complex) imag-part)
(put 'magnitude '(complex) magnitude)
(put 'angle '(complex) angle)
    \end{lstlisting}

    Describe in detail why this works. As an example, trace through all the
    procedures called in evaluating the expression \texttt{(magnitude z)} where
    \texttt{z} is the object shown in Figure 2.24. In particular, how many times
    is \texttt{apply-generic} invoked? What procedure is dispatched to in each
    case?
\end{quote}

The object shown in Figure 2.24 could be expressed as
\begin{lstlisting}
('complex . ('rectangular . (3 . 4)))
\end{lstlisting}

The procedures called in evaluating \texttt{(magnitude z)} in the corrected
system:
\begin{enumerate}
    \item \texttt{(apply-generic 'magnitude z)}: First application of
        \texttt{apply-generic}
    \item \texttt{(map type-tag z)}: Sets \texttt{type-tags} to
        \texttt{('complex)}
    \item \texttt{(get 'magnitude ('complex))}: This is what fails before the
        fix.
    \item \texttt{(apply magnitude ('rectangular . (3 . 4)))}: This is the
        generic \texttt{magnitude} defined by the complex package.
    \item \texttt{(apply-generic 'magnitude ('rectangular . (3 . 4)))}: Second
        application of \texttt{apply-generic}
    \item \texttt{(map type-tag ('rectangular . (3 . 4)))}: Sets
        \texttt{type-tags} to \texttt{('rectangular)}
    \item \texttt{get 'magnitude ('rectangular)}
    \item \texttt{(apply magnitude (3 . 4))}: This is the \texttt{magnitude}
        function from the rectangular package.
    \item \texttt{(sqrt (+ (square 3) (square 4)))}
\end{enumerate}

The function \texttt{magnitude} is being used at the level of abstraction at
which multiple types of numbers are defined. However, as it has been used so
far, it only really makes sense for complex numbers, and it is defined at that
level of abstraction. Packages are installed to provide implementations of
\texttt{magnitude} for multiple representations of complex numbers, but the
assumption that \texttt{magnitude} will be applied to complex numbers is
implicit. The definition of \texttt{z} makes that assumption explicit by adding
a \texttt{'complex} tag.

Alyssa's fix works by raising the complex-number primitives to the level of
abstraction at which multiple types of numbers are defined. When one of the
generic complex-number selectors is called with a \texttt{'complex}-tagged
argument, the outer tag is removed and the same \texttt{magnitude} is called
again on the contents of the argument, now exposing the tag indicating the
representation of complex numbers in use. Because there are two layers of tags,
\texttt{apply-generic} must be called twice to find the real implementation.

\section{Exercise 2.78}
\begin{quote}
    The internal procedures in the \texttt{scheme-number} package are
    essentially nothing more than calls to the primitive procedures \texttt{+},
    \texttt{-}, etc. It was not possible to use the primitives of the language
    directly because our type-tag system requires that each data object have a
    type attached to it. In fact, however, all Lisp implementations do have a
    type system, which they use internally. Primitive predicates such as
    \texttt{symbol?} and \texttt{number?} determine whether data objects have
    particular types. Modify the definitions of \texttt{type-tag},
    \texttt{contents}, and \texttt{attach-tag} from 2.4.2 so that our generic
    system takes advantage of Scheme's internal type system. That is to say,
    the system should work as before except that ordinary numbers should be
    represented simply as Scheme numbers rather than as pairs whose car is the
    symbol scheme-number.
\end{quote}

\lstinputlisting[firstline=186,lastline=201]{ch2/ex2.78.scm}

\end{document}
