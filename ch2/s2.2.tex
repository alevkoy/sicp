\documentclass{article}
\author{Abraham Levkoy}
\title{SICP exercises, section 2.2}

\usepackage{mystyle}

\begin{document}
\maketitle

\section{Exercise 2.17}
\begin{quote}
    Define a procedure \texttt{last-pair} that returns the list that contains
    only the last element of a given (nonempty) list:
    \begin{lstlisting}
(last-pair (list 23 72 149 34))
(34)
    \end{lstlisting}
\end{quote}

\lstinputlisting[firstline=3,lastline=6]{ch2/ex2.17.scm}

\section{Exercise 2.18}
\begin{quote}
    Define a procedure reverse that takes a list as argument and returns a list
    of the same elements in reverse order:
    \begin{lstlisting}
(reverse (list 1 4 9 16 25))
(25 16 9 4 1)
    \end{lstlisting}
\end{quote}

\lstinputlisting[firstline=3,lastline=9]{ch2/ex2.18.scm}

Somewhat atypically, this was much easier (for me, at least) to define as an
iterative process than as a recursive process. The iterative process builds the
result up from the end using \texttt{cons}. The only recursive implementation
that I came up with starts from the beginning of the resulting list, so it
needs to traverse the entire list that it already built on every recursive
invocation, which I did using the previously defined \texttt{append} procedure.

\section{Exercise 2.19}
\begin{quote}
    [Long discussion of using \texttt{cc} with multiple lists of coin
    denominations.]

    Define the procedures \texttt{first-denomination},
    \texttt{except-first-denomination} and \texttt{no-more?} in terms of
    primitive operations on list structures. Does the order of the list
    \texttt{coin-values} affect the answer produced by \texttt{cc}? Why or why
    not?
\end{quote}

\lstinputlisting[firstline=20,lastline=27]{ch2/ex2.19.scm}

The order of \texttt{coin-values} does not affect the answer produced. It just
affects the order in which the possible combinations are explored. At each level
of the recursion, there are still a branch in which one coin is excluded and a
branch in which the amount to be matched is decreased by the value of the
current coin. As long as the recursion continues to the full depth in which all
coins have been considered, all possible combinations will be explored and
evaluated.

\section{Exercise 2.20}
\begin{quote}
    \ldots[W]rite a procedure \texttt{same-parity} that takes one or more
    integers and returns a list of all the arguments that have the same
    even-odd parity as the first argument. For example,
    \begin{lstlisting}
(same-parity 1 2 3 4 5 6 7)
(1 3 5 7)

(same-parity 2 3 4 5 6 7)
(2 4 6)
    \end{lstlisting}
\end{quote}

\lstinputlisting[firstline=3,lastline=19]{ch2/ex2.20.scm}

This produces a recursive process, whereas I'd prefer an iterative one.
However, it allows the resulting list to be built up one element at a time
without traversing to the end to add each element. An iterative process that
considered successive elements of the input list, starting at the head, would
require repeated traversal of the output list to get the elements in the correct
order. This would involve a recursive component and probably more copying of
data.

\section{Exercise 2.21}
\begin{quote}
    The procedure \texttt{square-list} takes a list of numbers as argument and
    returns a list of the squares of those numbers.
    \begin{lstlisting}
(square-list (list 1 2 3 4))
(1 4 9 16)
    \end{lstlisting}
    Here are two different definitions of \texttt{square-list}. Complete both
    of them by filling in the missing expressions:
    \begin{lstlisting}
(define (square-list items)
  (if (null? items)
      nil
      (cons <??> <??>)))

(define (square-list items)
  (map <??> <??>))
    \end{lstlisting}
\end{quote}

\lstinputlisting[firstline=5,lastline=9]{ch2/ex2.21.scm}
\lstinputlisting[firstline=14,lastline=16]{ch2/ex2.21.scm}

\section{Exercise 2.22}
\begin{quote}
    Louis Reasoner tries to rewrite the first \texttt{square-list} procedure of
    Exercise 2.21 so that it evolves an iterative process:
    \begin{lstlisting}
(define (square-list items)
  (define (iter things answer)
    (if (null? things)
        answer
        (iter (cdr things)
              (cons (square (car things))
                    answer))))
  (iter items nil))
    \end{lstlisting}
    Unfortunately, defining \texttt{square-list} this way produces the answer
    list in the reverse order of the one desired. Why?
\end{quote}

He iterates through the input list forward, that is from head to tail, but he
builds the output list from tail to head. Because of this, the head of the
input is transformed into the tail of the output, and so on.

\begin{quote}
    Louis then tries to fix his bug by interchanging the arguments to cons:
    \begin{lstlisting}
(define (square-list items)
  (define (iter things answer)
    (if (null? things)
        answer
        (iter (cdr things)
              (cons answer
                    (square
                     (car things))))))
  (iter items nil))
    \end{lstlisting}
This doesn’t work either. Explain.
\end{quote}

This gets the output in basically the right order, but the structure is wrong. A
flat list is a sequence of pairs in which the \texttt{cdr} of each pair points
to the next pair, and the last \texttt{cdr} points to \texttt{nil}. By switching
the order of the arguments to \texttt{cons}, this code creates a pair in which
the \texttt{car} is a compound object, and the \texttt{cdr} is a bare integer.
The overall structure produced is a tree that is tited toward the left as
opposed to one tilted toward the right.

\section{Exercise 2.23}
\begin{quote}
    The procedure \texttt{for-each} is similar to \texttt{map}. It takes as
    arguments a procedure and a list of elements. However, rather than forming
    a list of the results, \texttt{for-each} just applies the procedure to each
    of the elements in turn, from left to right. The values returned by
    applying the procedure to the elements are not used at all ---
    \texttt{for-each} is used with procedures that perform an action, such as
    printing. The value returned by the call to \texttt{for-each} (not
    illustrated above) can be something arbitrary, such as \texttt{true}. Give
    an implementation of \texttt{for-each}.
\end{quote}

\lstinputlisting[firstline=3,lastline=9]{ch2/ex2.23.scm}

\section{Exercise 2.24}
\begin{quote}
    Suppose we evaluate the expression \texttt{(list 1 (list 2 (list 3 4)))}.
    Give the result printed by the interpreter, the corresponding
    box-and-pointer structure, and the interpretation of this as a tree (as in
    Figure 2.6).
\end{quote}

The interpreter gives the result
\begin{lstlisting}
(1 (2 (3 4)))
\end{lstlisting}

\begin{tikzpicture}[
        box/.style={rectangle, draw=black, thick, minimum size=5mm},
        % Found at http://tex.stackexchange.com/a/269410
        ptr/.style={{Circle[black, length=4pt]}-Latex,shorten <= -2pt}]
    % Nodes
    \node[box] (car1)                     { }; % (1
    \node[box] (cdr1) [right=0cm of car1] { };
    \node[box] (val1) [below=of car1]     {1};
    \node[box] (car2) [right=of cdr1]     { }; %    (2
    \node[box] (cdr2) [right=0cm of car2] { };
    \node[box] (val2) [below=of car2]     {2};
    \node[box] (car3) [right=of cdr2]     { }; %       (3
    \node[box] (cdr3) [right=0cm of car3] { };
    \node[box] (val3) [below=of car3]     {3};
    \node[box] (car4) [right=of cdr3]     { }; %           4)))
    \node[box] (cdr4) [right=0cm of car4] {/};
    \node[box] (val4) [below=of car4]     {4};

    % Lines
    \draw[ptr] (car1.center) -- (val1.north); % (1
    \draw[ptr] (cdr1.center) -- (car2.west);
    \draw[ptr] (car2.center) -- (val2.north); %    (2
    \draw[ptr] (cdr2.center) -- (car3.west);
    \draw[ptr] (car3.center) -- (val3.north); %       (3
    \draw[ptr] (cdr3.center) -- (car4.west);
    \draw[ptr] (car4.center) -- (val4.north); %          4)))
\end{tikzpicture}

\begin{allintypewriter}

\Tree [.{(1 (2 (3 4)))}
    1
    [.{(2 (3 4))}
        2
        [.{(3 4)}
            3
            4 ]]]

\end{allintypewriter}

\section{Exercise 2.25}
\begin{quote}
    Give combinations of \texttt{cars} and \texttt{cdrs} that will pick 7 from
    each of the following lists:

    \begin{lstlisting}
(1 3 (5 7) 9)
((7))
(1 (2 (3 (4 (5 (6 7))))))
    \end{lstlisting}
\end{quote}

\begin{enumerate}
    \item The sequence is \texttt{cdr, cdr, car, cdr, car}, i.e.
        \begin{lstlisting}
(car (cdr (car (cdr (cdr x )))))
        \end{lstlisting}
        \texttt{cdr} produces a pair, so the final \texttt{car} is necessary to
        remove the \texttt{7} from \texttt{(7)}.
    \item \texttt{(car (car x))}
    \item Each \texttt{cdr} produces a pair, of which the first element is
        another list, so \texttt{car} must be used to access that list directly.
        There are six layers of nested lists, so there need to be 6
        \texttt{cdr}/\texttt{car} pairs.
        \begin{lstlisting}
(car (cdr
    (car (cdr
        (car (cdr
            (car (cdr
                (car (cdr
                    (car (cdr x))))))))))))
        \end{lstlisting}
\end{enumerate}

\section{Exercise 2.26}
\begin{quote}
	Suppose we define x and y to be two lists:
    \begin{lstlisting}
(define x (list 1 2 3))
(define y (list 4 5 6))
    \end{lstlisting}
	What result is printed by the interpreter in response to evaluating each of
	the following expressions:
    \begin{lstlisting}
(append x y)
(cons x y)
(list x y)
    \end{lstlisting}
\end{quote}

\begin{enumerate}
    \item \texttt{append} concatenates two lists; the result is a list:
        \texttt{(1 2 3 4 5 6)}.
    \item \texttt{cons} constructs a pair out of its arguments, effectively
        prepending the first argument to the sceond argument if the second
        argument is a list: \texttt{((1 2 3) 4 5 6)}.
    \item \texttt{list} constructs a list in which each argument is a single
        element: \texttt{((1 2 3) (4 5 6))}.
\end{enumerate}

\section{Exercise 2.27}
\begin{quote}
    Modify your reverse procedure of Exercise 2.18 to produce a
    \texttt{deep-reverse} procedure that takes a list as argument and returns
    as its value the list with its elements reversed and with all sublists
    deep-reversed as well. For example,
    \begin{lstlisting}
(define x
  (list (list 1 2) (list 3 4)))

x
((1 2) (3 4))

(reverse x)
((3 4) (1 2))

(deep-reverse x)
((4 3) (2 1))
    \end{lstlisting}
\end{quote}

\lstinputlisting[firstline=3,lastline=13]{ch2/ex2.27.scm}

\section{Exercise 2.28}
\begin{quote}
    Write a procedure \texttt{fringe} that takes as argument a tree
    (represented as a list) and returns a list whose elements are all the
    leaves of the tree arranged in left-to-right order. For example,
    \begin{lstlisting}
(define x
  (list (list 1 2) (list 3 4)))

(fringe x)
(1 2 3 4)

(fringe (list x x))
(1 2 3 4 1 2 3 4)
    \end{lstlisting}
\end{quote}

\lstinputlisting[firstline=3,lastline=16]{ch2/ex2.28.scm}

\section{Exercise 2.29}
\begin{quote}
	[Long explanation of binary mobiles.]

    \begin{enumerate}
        \item Write the corresponding selectors \texttt{left-branch} and
            \texttt{right-branch}, which return the branches of a mobile, and
            \texttt{branch-length} and \texttt{branch-structure}, which return
            the components of a branch.
        \item Using your selectors, define a procedure \texttt{total-weight}
            that returns the total weight of a mobile.
        \item A mobile is said to be \emph{balanced} if the torque applied by
            its top-left branch is equal to that applied by its top-right
            branch (that is, if the length of the left rod multiplied by the
            weight hanging from that rod is equal to the corresponding product
            for the right side) and if each of the submobiles hanging off its
            branches is balanced. Design a predicate that tests whether a
            binary mobile is balanced.
        \item Suppose we change the representation of mobiles so that the
            constructors are
            \begin{lstlisting}
(define (make-mobile left right)
  (cons left right))

(define (make-branch length structure)
  (cons length structure))
            \end{lstlisting}
            How much do you need to change your programs to convert to the new
            representation?
    \end{enumerate}
\end{quote}

\begin{enumerate}
    \item \lstinputlisting[firstline=11,lastline=21]{ch2/ex2.29.scm}

    \item \lstinputlisting[firstline=40,lastline=52]{ch2/ex2.29.scm}

    \item \lstinputlisting[firstline=75,lastline=87]{ch2/ex2.29.scm}

    \item The preceding functions are written in terms of \texttt{make-mobile}
        and \texttt{make-branch}. They do not need to be changed at all if the
        underlying representation of mobiles or branches changes. One assumption
        I do make about the representation is that a mobile consisting of just
        one weight is represented using a bare numeric value.
\end{enumerate}

\section{Exercise 2.30}
\begin{quote}
    Define a procedure \texttt{square-tree} analogous to the
    \texttt{square-list} procedure of Exercise 2.21. That is,
    \texttt{square-tree} should behave as follows:
    \begin{lstlisting}
(square-tree
 (list 1
       (list 2 (list 3 4) 5)
       (list 6 7)))
(1 (4 (9 16) 25) (36 49))
    \end{lstlisting}
    Define \texttt{square-tree} both directly (i.e., without using any
    higher-order procedures) and also by using \texttt{map} and recursion.
\end{quote}

Definition without higher-order procedures:
\lstinputlisting[firstline=6,lastline=10]{ch2/ex2.30.scm}

Definition with \texttt{map}:
\lstinputlisting[firstline=21,lastline=26]{ch2/ex2.30.scm}

\section{Exercise 2.31}
\begin{quote}
    Abstract your answer to Exercise 2.30 to produce a procedure
    \texttt{tree-map} with the property that \texttt{square-tree} could be
    defined as
    \begin{lstlisting}
(define (square-tree tree)
  (tree-map square tree))
    \end{lstlisting}
\end{quote}

This is a simple matter of parameterizing the \texttt{map}-based definition
with an operation to perform on each leaf:
\lstinputlisting[firstline=34,lastline=39]{ch2/ex2.30.scm}

\section{Exercise 2.32}
\begin{quote}
    We can represent a set as a list of distinct elements, and we can represent
    the set of all subsets of the set as a list of lists. For example, if the
    set is \texttt{(1 2 3)}, then the set of all subsets is \texttt{(() (3) (2)
    (2 3) (1) (1 3) (1 2) (1 2 3))}. Complete the following definition of a
    procedure that generates the set of subsets of a set and give a clear
    explanation of why it works:
    \begin{lstlisting}
(define (subsets s)
  (if (null? s)
      (list nil)
      (let ((rest (subsets (cdr s))))
        (append rest (map <??> rest)))))
    \end{lstlisting}
\end{quote}

\lstinputlisting[firstline=4,lastline=11]{ch2/ex2.32.scm}

The operation for the mapping is to prepend the first element of the list under
consideration to each of the subsets of the list without that first element.

This works, because the subsets of a set are the subsets of a set can be
divided into those not containg the first element and those containing the
first element. Those not containing the first element have already been
collected in \texttt{rest}. Those containing the first element can be produced
by prepending the first element to each of the subsets that does not have it.
(It could also be appended, but this would be more work.)

When these two groups are combined, the complete set of subsets is formed. Thus,
the recursively defined procedure works by progressively building up the sets of
subsets represented by longer lists, starting with the empty list in the base
case and ending with the full list of elements in the input set.

\section{Exercise 2.33}
\begin{quote}
    Fill in the missing expressions to complete the following definitions of
    some basic list-manipulation operations as accumulations:
    \begin{lstlisting}
(define (map p sequence)
  (accumulate (lambda (x y) <??>)
              nil sequence))

(define (append seq1 seq2)
  (accumulate cons <??> <??>))

(define (length sequence)
  (accumulate <??> 0 sequence))
    \end{lstlisting}
\end{quote}

\texttt{map} is pretty straightforward. We perform the operation on the first
element and concatentate it with the list of already-transformed elements.
\lstinputlisting[firstline=12,lastline=16]{ch2/ex2.33.scm}

\texttt{append}, as usual, ends up dismantling its first parameter entirely and
rebuilding it at the front of its scond.
\lstinputlisting[firstline=22,lastline=23]{ch2/ex2.33.scm}

\texttt{length} may be the most typical accumulation in that it just adds up a
number. The only weird thing is that it does not actually examine the elements
of the input.
\lstinputlisting[firstline=29,lastline=33]{ch2/ex2.33.scm}

\section{Exercise 2.34}
\begin{quote}
	[Explanation of Horner's rule.]

    Fill in the following template to produce a procedure that evaluates a
    polynomial using Horner's rule. Assume that the coefficients of the
    polynomial are arranged in a sequence, from $a_0$ through $a_n$.
    \begin{lstlisting}
(define
  (horner-eval x coefficient-sequence)
  (accumulate
   (lambda (this-coeff higher-terms)
     <??>)
   0
   coefficient-sequence))
    \end{lstlisting}
For example, to compute $1+3x+5x^3+x^5$ at $x=2$ you would evaluate
    \begin{lstlisting}
(horner-eval 2 (list 1 3 0 5 0 1))
    \end{lstlisting}
\end{quote}

\lstinputlisting[firstline=41,lastline=46]{ch2/ex2.33.scm}

\section{Exercise 2.35}
\begin{quote}
	Redefine \texttt{count-leaves} from 2.2.2 as an accumulation:

	\begin{lstlisting}
(define (count-leaves t)
  (accumulate <??> <??> (map <??> <??>)))
	\end{lstlisting}
\end{quote}

\texttt{map} does the heavy lifting here. I could also imagine a version in
which the function supplied to \texttt{accumulate} handles the different cases,
and the input sequence is the tree itself.
\lstinputlisting[firstline=55,lastline=60]{ch2/ex2.33.scm}

\section{Exercise 2.36}
\begin{quote}
    The procedure \texttt{accumulate-n} is similar to \texttt{accumulate}
    except that it takes as its third argument a sequence of sequences, which
    are all assumed to have the same number of elements. It applies the
    designated accumulation procedure to combine all the first elements of the
    sequences, all the second elements of the sequences, and so on, and returns
    a sequence of the results. For instance, if \texttt{s} is a sequence
    containing four sequences, \texttt{((1 2 3) (4 5 6) (7 8 9) (10 11 12))},
    then the value of \texttt{(accumulate-n + 0 s)} should be the sequence
    \texttt{(22 26 30)}. Fill in the missing expressions in the following
    definition of \texttt{accumulate-n}:
    \begin{lstlisting}
(define (accumulate-n op init seqs)
  (if (null? (car seqs))
      nil
      (cons (accumulate op init <??>)
            (accumulate-n op init <??>))))
    \end{lstlisting}
\end{quote}

The first element of the results list will be an accumulation across the first
elements of the input lists. The rest of the results list will be an
\texttt{accumulate-n} call on the input lists with the first elements removed.
\lstinputlisting[firstline=73,lastline=77]{ch2/ex2.33.scm}

\section{Exercise 2.37}
\begin{quote}
    [Long explanation of matrix operations.]

    Fill in the missing expressions in the following procedures for computing
    the other matrix operations. (The procedure \texttt{accumulate-n} is
    defined in Exercise 2.36.)
    \begin{lstlisting}
(define (matrix-*-vector m v)
  (map <??> m))

(define (transpose mat)
  (accumulate-n <??> <??> mat))

(define (matrix-*-matrix m n)
  (let ((cols (transpose n)))
    (map <??> m)))
    \end{lstlisting}
\end{quote}

The representation of matrices in this problem would suggest that vectors are
taken to be row vectors, i.e. with one row. However, matrix-vector
multiplication is only defined for column vectors, so vectors in the following
code are taken to be column vectors.

\lstinputlisting[firstline=23,lastline=26]{ch2/ex2.37.scm}
I didn't expect \texttt{transpose} to be this elegant, but there is a
transposition inherent in \texttt{accumulate-n}.
\lstinputlisting[firstline=38,lastline=39]{ch2/ex2.37.scm}
\texttt{matrix-*-matrix} is somewhat more elaborate.
\lstinputlisting[firstline=46,lastline=52]{ch2/ex2.37.scm}

\section{Exercise 2.38}
\begin{quote}
    The \texttt{accumulate} procedure is also known as \texttt{fold-right},
    because it combines the first element of the sequence with the result of
    combining all the elements to the right. There is also a
    \texttt{fold-left}, which is similar to \texttt{fold-right}, except that it
    combines elements working in the opposite direction:
    \begin{lstlisting}
(define (fold-left op initial sequence)
  (define (iter result rest)
    (if (null? rest)
        result
        (iter (op result (car rest))
              (cdr rest))))
  (iter initial sequence))
    \end{lstlisting}
    What are the values of
    \begin{lstlisting}
(fold-right / 1 (list 1 2 3))
(fold-left  / 1 (list 1 2 3))
(fold-right list nil (list 1 2 3))
(fold-left  list nil (list 1 2 3))
    \end{lstlisting}
    Give a property that op should satisfy to guarantee that
    \texttt{fold-right} and \texttt{fold-left} will produce the same values for
    any sequence.
\end{quote}

\begin{lstlisting}
(fold-right / 1 (list 1 2 3))
> 3/2
(fold-left  / 1 (list 1 2 3))
> 1/6
(fold-right list nil (list 1 2 3))
> (1 (2 (3 ())))
(fold-left  list nil (list 1 2 3))
> (((() 1) 2) 3)
\end{lstlisting}

In order for \texttt{fold-left} and \texttt{fold-right} to produce the same
result, \texttt{op} must be commutative, i.e. the order of \texttt{op}'s
operands must not affect the result.

\section{Exercise 2.39}
\begin{quote}
    Complete the following definitions of \texttt{reverse} (Exercise 2.18) in
    terms of \texttt{fold-right} and \texttt{fold-left} from Exercise 2.38:
    \begin{lstlisting}
(define (reverse sequence)
  (fold-right
   (lambda (x y) <??>) nil sequence))

(define (reverse sequence)
  (fold-left
   (lambda (x y) <??>) nil sequence))
    \end{lstlisting}
\end{quote}

\lstinputlisting[firstline=4,lastline=8]{ch2/ex2.39.scm}
The \texttt{fold-left} version can be written using \texttt{append} very
analogously to the \texttt{fold-right} version, but \texttt{cons} is more
efficient.
\lstinputlisting[firstline=13,lastline=17]{ch2/ex2.39.scm}

\section{Exercise 2.40}
\begin{quote}
    Define a procedure \texttt{unique-pairs} that, given an integer $n$,
    generates the sequence of pairs $(i,j)$ with $1 \leq j < i \leq n$. Use
    \texttt{unique-pairs} to simplify the definition of
    \texttt{prime-sum-pairs} given above.
\end{quote}

This solution is pretty much copying and pasting from the text.
\lstinputlisting[firstline=15,lastline=20]{ch2/ex2.40.scm}
\lstinputlisting[firstline=51,lastline=54]{ch2/ex2.40.scm}

\section{Exercise 2.41}
\begin{quote}
    Write a procedure to find all ordered triples of distinct positive integers
    $i$, $j$, and $k$ less than or equal to a given integer $n$ that sum to a given
    integer $s$.
\end{quote}

This is pretty analogous to \texttt{prime-sum-pairs}.
\lstinputlisting[firstline=62,lastline=80]{ch2/ex2.40.scm}

\section{Exericse 2.42}
\begin{quote}
    [Explanation of $n$-queens problem.]

    We implement this solution as a procedure \texttt{queens}, which returns a
    sequence of all solutions to the problem of placing $n$ queens on an $n
    \times n$ chessboard.  \texttt{Queens} has an internal procedure
    \texttt{queen-cols} that returns the sequence of all ways to place queens
    in the first $k$ columns of the board.

    \begin{lstlisting}
(define (queens board-size)
  (define (queen-cols k)
    (if (= k 0)
        (list empty-board)
        (filter
         (lambda (positions)
           (safe? k positions))
         (flatmap
          (lambda (rest-of-queens)
            (map (lambda (new-row)
                   (adjoin-position
                    new-row
                    k
                    rest-of-queens))
                 (enumerate-interval
                  1
                  board-size)))
          (queen-cols (- k 1))))))
  (queen-cols board-size))
    \end{lstlisting}

    In this procedure \texttt{rest-of-queens} is a way to place $k - 1$ queens
    in the first $k - 1$ columns, and \texttt{new-row} is a proposed row in
    which to place the queen for the $k^\textrm{th}$ column. Complete the
    program by implementing the representation for sets of board positions,
    including the procedure \texttt{adjoin-position}, which adjoins a new
    row-column position to a set of positions, and \texttt{empty-board}, which
    represents an empty set of positions. You must also write the procedure
    \texttt{safe?}, which determines for a set of positions, whether the queen
    in the $k^\textrm{th}$ column is safe with respect to the others. (Note
    that we need only check whether the new queen is safe -- the other queens
    are already guaranteed safe with respect to each other.)
\end{quote}

The representation of board positions on an $n \times n$ board is a list of
columns where each column is represented as the row number on which a queen is
placed, and the $n^\textrm{th}$ column is the head of the list. (The orientation
of the list doesn't actually matter, since the horizontal reflection of an
$n$-queens solution is also a solution.) Row numbers are assumed (but not
verified) to be within the bounds of the board that they represent.
\lstinputlisting[firstline=30,lastline=53]{ch2/ex2.42.scm}

\section{Exercise 2.43}
\begin{quote}
    Louis Reasoner is having a terrible time doing Exercise 2.42. His
    \texttt{queens} procedure seems to work, but it runs extremely slowly.
    (Louis never does manage to wait long enough for it to solve even the $6
    \times 6$ case.) When Louis asks Eva Lu Ator for help, she points out that
    he has interchanged the order of the nested mappings in the
    \texttt{flatmap}, writing it as

    \begin{lstlisting}
(flatmap
 (lambda (new-row)
   (map (lambda (rest-of-queens)
          (adjoin-position
           new-row k rest-of-queens))
        (queen-cols (- k 1))))
 (enumerate-interval 1 board-size))
    \end{lstlisting}

    Explain why this interchange makes the program run slowly. Estimate how
    long it will take Louis's program to solve the eight-queens puzzle,
    assuming that the program in Exercise 2.42 solves the puzzle in time $T$.
\end{quote}

The code in Exercise 2.42, at each level of recursion, generates the solution to
the level below it once. Louis Reasoner's code generates the solution to the
previous level (recursively) $k$ times at level $k$.

Suppose that $T$ is the sum of the amount of time it takes to solve each level
of the recursion, $$T = T_n + T_{n-1} + \cdots + T_1$$ At each level $k$, Louis'
solution computes the solution for $(k-1)$ $k$ times. Thus,
\begin{IEEEeqnarray*}{rCl}
    T_{\textrm{Louis}} & = & T_n + nT_{n-1} + n(n-1)T_{n-2} + \cdots +
                             n(n-1)(n-2) \cdots 2 \cdot 1T_1 \\
                       & = & O(T_n + nT_{n-1} + \cdots + n^nT_1) \\
\end{IEEEeqnarray*}

Supposing that $T_k \ll O(k^k)$, the exponential factor in the previous
expression dominates. (This seems reasonable, since the code for Exercise 2.42
finished quickly.) Thus, $T_{\textrm{Louis}} = O(n^nT)$.

\section{Exercise 2.44}
\begin{quote}
    Define the procedure \texttt{up-split} used by \texttt{corner-split}. It is
    similar to \texttt{right-split}, except that it switches the roles of
    \texttt{below} and \texttt{beside}.
\end{quote}

\emph{MIT Scheme does not currently support the picture language described in
\emph{SICP}. Consequently, while I will attempt to produce reasonable solutions
to the problems that deal with the picture language, it is not possible for me
to truly produce working solutions.}

\lstinputlisting[firstline=15,lastline=21]{ch2/ex2.44.scm}

\section{Exercise 2.45}
\begin{quote}
    \texttt{Right-split} and \texttt{up-split} can be expressed as instances of
    a general splitting operation. Define a procedure split with the property
    that evaluating
    \begin{lstlisting}
(define right-split (split beside below))
(define up-split (split below beside))
    \end{lstlisting}
    produces procedures \texttt{right-split} and \texttt{up-split} with the same
    behaviors as the ones already defined.
\end{quote}

\lstinputlisting[firstline=25,lastline=32]{ch2/ex2.44.scm}

\section{Exercise 2.46}
\begin{quote}
    A two-dimensional vector $v$ running from the origin to a point can be
    represented as a pair consisting of an $x$-coordinate and a $y$-coordinate.
    Implement a data abstraction for vectors by giving a constructor
    \texttt{make-vect} and corresponding selectors \texttt{xcor-vect} and
    \texttt{ycor-vect}. In terms of your selectors and constructor, implement
    procedures \texttt{add-vect}, \texttt{sub-vect}, and \texttt{scale-vect}
    that perform the operations vector addition, vector subtraction, and
    multiplying a vector by a scalar:
    \begin{IEEEeqnarray*}{rCl}
        (x_1, y_1) + (x_2, y_2) & = & (x_1 + x_2, y_1 + y_2), \\
        (x_1, y_1) - (x_2, y_2) & = & (x_1 - x_2, y_1 - y_2), \\
        s \cdot (x, y) & = & (sx, sy). \\
    \end{IEEEeqnarray*}
\end{quote}

\lstinputlisting[firstline=39,lastline=62]{ch2/ex2.44.scm}

\section{Exercise 2.47}
\begin{quote}
    Here are two possible constructors for frames:
    \begin{lstlisting}
(define (make-frame origin edge1 edge2)
  (list origin edge1 edge2))

(define (make-frame origin edge1 edge2)
  (cons origin (cons edge1 edge2)))
    \end{lstlisting}
    For each constructor supply the appropriate selectors to produce an
    implementation for frames.
\end{quote}

Implementation 1:
\lstinputlisting[firstline=80,lastline=87]{ch2/ex2.44.scm}

Implementation 2:
\lstinputlisting[firstline=108,lastline=116]{ch2/ex2.44.scm}

\section{Exercise 2.48}
\begin{quote}
    A directed line segment in the plane can be represented as a pair of
    vectors --- the vector running from the origin to the start-point of the
    segment, and the vector running from the origin to the end-point of the
    segment. Use your vector representation from Exercise 2.46 to define a
    representation for segments with a constructor \texttt{make-segment} and
    selectors \texttt{start-segment} and end-segment.
\end{quote}

\lstinputlisting[firstline=131,lastline=138]{ch2/ex2.44.scm}

\section{Exercise 2.49}
\begin{quote}
    Use \texttt{segments->painter} to define the following primitive painters:
    \begin{itemize}
        \item The painter that draws the outline of the designated frame.
        \item The painter that draws an “X” by connecting opposite corners of
            the frame.
        \item The painter that draws a diamond shape by connecting
            the midpoints of the sides of the frame.
        \item The \texttt{wave} painter.
    \end{itemize}
\end{quote}

\lstinputlisting[firstline=177,lastline=201]{ch2/ex2.44.scm}

I have skipped the \texttt{wave} painter, because 1.) it would be very tedious
to define, 2.) there does not appear to be a single, recognized solution, and
3.) I have no good way to verify that a proposed solution is even approximately
correct.

\end{document}
