\documentclass{article}
\author{Abraham Levkoy}
\title{SICP exercises, section 2.3}

\usepackage{mystyle}

\begin{document}
\maketitle

\section{Exercise 2.53}
\begin{quote}
    What would the interpreter print in response to evaluating each of the
    following expressions?
\end{quote}

\begin{lstlisting}
(list 'a 'b 'c)
> (a b c)
(list (list 'george))
> ((george))
(cdr '((x1 x2) (y1 y2)))
> ((y1 y2))
(cadr '((x1 x2) (y1 y2)))
> (y1 y2)
(pair? (car '(a short list)))
> #f
(memq 'red '((red shoes) (blue socks)))
> #f
(memq 'red '(red shoes blue socks))
> #t
\end{lstlisting}

\section{Exercise 2.54}
\begin{quote}
    Two lists are said to be \texttt{equal?} if they contain equal elements
    arranged in the same order. For example,

    \begin{lstlisting}
(equal? '(this is a list)
        '(this is a list))
    \end{lstlisting}

is true, but

    \begin{lstlisting}
(equal? '(this is a list)
        '(this (is a) list))
    \end{lstlisting}

    is false. To be more precise, we can define \texttt{equal?} recursively in
    terms of the basic \texttt{eq?} equality of symbols by saying that
    \texttt{a} and \texttt{b} are \texttt{equal?} if they are both symbols and
    the symbols are \texttt{eq?}, or if they are both lists such that
    \texttt{(car a)} is \texttt{equal?} to \texttt{(car b)} and \texttt{(cdr
    a)} is \texttt{equal?} to \texttt{(cdr b)}. Using this idea, implement
    \texttt{equal?} as a procedure.
\end{quote}

\lstinputlisting[firstline=3,lastline=9]{ch2/ex2.54.scm}

\section{Exercise 2.55}
\begin{quote}
    Eva Lu Ator types to the interpreter the expression

    \begin{lstlisting}
(car ''abracadabra)
    \end{lstlisting}

    To her surprise, the interpreter prints back \texttt{quote}. Explain.
\end{quote}

The character \texttt{'} is a syntactic shorthand for using the special form
\texttt{quote} with the following object as the argument. It evaluates to a
list of the symbols in the argument. When it is used on itself, as here, the
inner \texttt{quote} gets evaluated by the outer \texttt{quote}, producing a
list with \texttt{quote} as its first element, i.e.
\begin{lstlisting}
(quote abracadabra)
\end{lstlisting}
\texttt{car} then evaluates to the first element of its argument.

\section{Exercise 2.56}
\begin{quote}
    Show how to extend the basic differentiator to handle more kinds of
    expressions. For instance, implement the differentiation rule
    $$\frac{d(u^n)}{dx} = nu^{n-1}\frac{du}{dx}$$
    by adding a new clause to the \texttt{deriv} program and defining
    appropriate procedures \texttt{exponentiation?}, \texttt{base},
    \texttt{exponent}, and \texttt{make-exponentiation}. (You may use the
    symbol \texttt{**} to denote exponentiation.) Build in the rules that
    anything raised to the power 0 is 1 and anything raised to the power 1 is
    the thing itself.
\end{quote}

\lstinputlisting[firstline=47,lastline=60]{ch2/ex2.56.scm}
\lstinputlisting[firstline=63,lastline=87]{ch2/ex2.56.scm}

\end{document}
