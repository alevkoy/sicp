\documentclass{article}
\author{Abraham Levkoy}
\title{SICP exercises, section 1.1}

\usepackage{color}
\usepackage{listings}

\begin{document}
\maketitle

\lstset{
    language=Lisp,
    otherkeywords={define,if,else},
    backgroundcolor=\color{white},
    basicstyle=\ttfamily,
    keywordstyle=\color{blue}\ttfamily,
    stringstyle=\color{red}\ttfamily,
    commentstyle=\color{magenta}\\ttfamily
}

\section{Exercise 1.1}
\begin{quote}
    Below is a sequence of expressions. What is the result printed by the
    interpreter in response to each expression? Assume that the sequence is to
    be evaluated in the order in which it is presented.
\end{quote}

\noindent \lstinline|10| \\
\emph{10}

\noindent \lstinline|(+ 5 3 4)| \\
\emph{12}

\noindent \lstinline|(- 9 1)| \\
\emph{8}

\noindent \lstinline|(/ 6 2)| \\
\emph{3}

\noindent \lstinline|(+ (* 2 4) (- 4 6))| \\
\emph{6}

\noindent \lstinline|(define a 3)| \\
\emph{[stored value a]}

\noindent \lstinline|(define b (+ a 1))| \\
\emph{[stored value b]}

\noindent \lstinline|(+ a b (* a b))| \\
\emph{19}

\noindent \lstinline|(= a b)| \\
\emph{\#f}

\begin{lstlisting}
(if (and (> b a) (< b (* a b)))
    b
    a)
\end{lstlisting}
\emph{4}

\begin{lstlisting}
(cond ((= a 4) 6)
      ((= b 4) (+ 6 7 a))
      (else 25))
\end{lstlisting}
\emph{16}

\noindent \lstinline|(+ 2 (if (> b a) b a))| \\
\emph{6}

\begin{lstlisting}
(* (cond ((> a b) a)
         ((< a b) b)
         (else -1))
   (+ a 1))
\end{lstlisting}
\emph{16}

\section{Exercise 1.2}
\begin{quote}
    Translate the following expression into prefix form:
    \begin{math}
        \frac{5+4+(2-(3-6+\frac{4}{5}))}{3(6-2)(2-7)}
    \end{math}.
\end{quote}

\lstinline|(/ (+ 5 4 (- 2 (- 3 (+ 6 (/ 4 5))))) (* 3 (- 6 2) (- 2 7)))|

\section{Exercise 1.3}
\begin{quote}
	Define a procedure that takes three numbers as arguments and returns the sum
	of the squares of the two larger numbers.
\end{quote}

\lstinputlisting[firstline=2]{ch1/1.3.scm}

\section{Exercise 1.4}
\begin{quote}
	Observe that our model of evaluation allows for combinations whose operators
    are compound expressions. Use this observation to describe the behavior of
    the following procedure:
    \begin{lstlisting}
(define (a-plus-abs-b a b)
  ((if (> b 0) + -) a b))
    \end{lstlisting}
\end{quote}

If $b$ is positive, the procedure returns \lstinline|(+ a b)|;
otherwise, it returns \lstinline|(- a b)|. Thus, it returns $a + |b|$.

\end{document}
