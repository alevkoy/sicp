\documentclass{article}
\author{Abraham Levkoy}
\title{SICP exercises, section 1.3}

\usepackage{mystyle}

\begin{document}
\maketitle

\section{Exercise 1.29}
\begin{quote}
    Simpson's Rule is a more accurate method of numerical integration than the
    method illustrated above. Using Simpson's Rule, the integral of a function
    $f$ between $a$ and $b$ is approximated as
    \begin{equation*}
        \frac{h}{3}(y_0+4y_1+2y_2+4y_3+2y_4+\cdots+2y_{n-2}+4y_{n-1}+y_n),
    \end{equation*}
    where $h=(b−a)/n$, for some even integer $n$, and $y_k=f(a+kh)$.
    (Increasing $n$ increases the accuracy of the approximation.) Define a
    procedure that takes as arguments $f$, $a$, $b$, and $n$ and returns the
    value of the integral, computed using Simpson’s Rule. Use your procedure to
    integrate \texttt{cube} between 0 and 1 (with $n=100$ and $n=1000$), and
    compare the results to those of the \texttt{integral} procedure shown
    above.
\end{quote}

\lstinputlisting[firstline=4,lastline=26]{ch1/ex1.29.scm}

\begin{lstlisting}
> (integral cube 0 1 .01)
.24998750000000042
> (simpson cube 0 1 100)
.25
> (integral cube 0 1 .001)
.2467166666666667
> (simpson cube 0 1 1000)
.25
\end{lstlisting}

\texttt{simpson} finds the exact value of the integral ($\frac{1}{4}$) and is
thus more accurate than \texttt{integral}. In fact, it is spookily accurate. It
seems to produce results using only 2 intervals that are as accurate as
\texttt{integral} with 100 intervals.

\section{Exercise 1.30}
\begin{quote}
    The \texttt{sum} procedure above generates a linear recursion. The
    procedure can be rewritten so that the sum is performed iteratively. Show
    how to do this by filling in the missing expressions in the following
    definition:
    \begin{lstlisting}
(define (sum term a next b)
  (define (iter a result)
    (if <??>
        <??>
        (iter <??> <??>)))
  (iter <??> <??>))
    \end{lstlisting}
\end{quote}

\lstinputlisting[firstline=5,lastline=12]{ch1/ex1.30.scm}

This implementation produces the same results as the recursive implementation
when used in \texttt{simpson}.

\section{Exercise 1.31}
\begin{quote}
    \begin{enumerate}
        \item The \texttt{sum} procedure is only the simplest of a vast number
            of similar abstractions that can be captured as higher-order
            procedures. Write an analogous procedure called \texttt{product}
            that returns the product of the values of a function at points over
            a given range. Show how to define \texttt{factorial} in terms of
            \texttt{product}. Also use \texttt{product} to compute
            approximations to $\pi$ using the formula
            \begin{equation*}
                \frac{\pi}{4}=
                \frac{2\cdot4\cdot4\cdot6\cdot6\cdot8\cdot\cdots}
                     {3\cdot3\cdot5\cdot5\cdot7\cdot7\cdot\cdots}.
            \end{equation*}
        \item If your product procedure generates a recursive process, write
            one that generates an iterative process. If it generates an
            iterative process, write one that generates a recursive process.
    \end{enumerate}
\end{quote}

\subsection{Recursive process}
\lstinputlisting[firstline=4,lastline=11]{ch1/ex1.31.scm}

\lstinputlisting[firstline=13,lastline=24]{ch1/ex1.31.scm}

\lstinputlisting[firstline=31,lastline=42]{ch1/ex1.31.scm}
This procedure counts terms in a way such that there are two terms of this
procedure for every one term of the canonical formulation of the Wallis Formula.
I suppose my procedure is half as precise as one using the canonical enumeration
of terms. No matter: I also do half as much work per term, so I can just use
twice as many terms.

\subsection{Iterative process}
\lstinputlisting[firstline=52,lastline=62]{ch1/ex1.31.scm}

\section{Exercise 1.32}
\begin{quote}
    \begin{enumerate}
        \item Show that \texttt{sum} and \texttt{product} (Exercise 1.31) are
            both special cases of a still more general notion called
            \texttt{accumulate} that combines a collection of terms, using some
            general accumulation function:
            \begin{lstlisting}
(accumulate
 combiner null-value term a next b)
            \end{lstlisting}
            \texttt{Accumulate} takes as arguments the same term and range
            specifications as \texttt{sum} and \texttt{product}, together with
            a \texttt{combiner} procedure (of two arguments) that specifies how
            the current term is to be combined with the accumulation of the
            preceding terms and a \texttt{null-value} that specifies what base
            value to use when the terms run out. Write \texttt{accumulate} and
            show how \texttt{sum} and \texttt{product} can both be defined as
            simple calls to \texttt{accumulate}.
        \item If your \texttt{accumulate} procedure generates a recursive
            process, write one that generates an iterative process. If it
            generates an iterative process, write one that generates a
            recursive process.
    \end{enumerate}
\end{quote}

\subsection{Recursive process}
\lstinputlisting[firstline=4,lastline=13]{ch1/ex1.32.scm}

\subsection{Iterative process}
\lstinputlisting[firstline=43,lastline=54]{ch1/ex1.32.scm}

\section{Exercise 1.33}
\begin{quote}
    You can obtain an even more general version of \texttt{accumulate}
    (Exercise 1.32) by introducing the notion of a filter on the terms to be
    combined. That is, combine only those terms derived from values in the
    range that satisfy a specified condition. The resulting
    \texttt{filtered-accumulate} abstraction takes the same arguments as
    \texttt{accumulate}, together with an additional predicate of one argument
    that specifies the filter. Write \texttt{filtered-accumulate} as a
    procedure. Show how to express the following using
    \texttt{filtered-accumulate}:
    \begin{enumerate}
        \item the sum of the squares of the prime numbers in the interval $a$
            to $b$ (assuming that you have a \texttt{prime?} predicate already
            written)
        \item the product of all the positive integers less than $n$ that are
            relatively prime to $n$ (i.e., all positive integers $i<n$ such that
            $\textrm{GCD}(i,n)=1)$.
    \end{enumerate}
\end{quote}

\lstinputlisting[firstline=4,lastline=21]{ch1/ex1.33.scm}

\subsection{Sum of squares of prime numbers in interval $a$ to $b$}
\lstinputlisting[firstline=44,lastline=55]{ch1/ex1.33.scm}

\subsection{Product of positive integers less than and relatively prime to $n$}
\lstinputlisting[firstline=57,lastline=70]{ch1/ex1.33.scm}

\section{Exercise 1.34}
\begin{quote}
    Suppose we define the procedure
    \begin{lstlisting}
(define (f g) (g 2))
    \end{lstlisting}
    Then we have
    \begin{lstlisting}
(f square)
4
(f (lambda (z) (* z (+ z 1))))
6
    \end{lstlisting}
    What happens if we (perversely) ask the interpreter to evaluate the
    combination \texttt{(f f)}? Explain.
\end{quote}

\texttt{(f f)} evaluates to \texttt{(f 2)}, which in turn evaluates to
\texttt{(2 2)}. \texttt{2} isn't the name of a procedure, so this produces an
interpreter error.

\section{Exercise 1.35}
\begin{quote}
    Show that the golden ratio $\phi$ (1.2.2) is a fixed point of the
    transformation $x\rightarrow1+1/x$, and use this fact to compute $\phi$ by
    means of the fixed-point procedure.
\end{quote}

\lstinputlisting[firstline=19,lastline=22]{ch1/ex1.35.scm}

\section{Exercise 1.36}
\begin{quote}
    Modify \texttt{fixed-point} so that it prints the sequence of
    approximations it generates, using the \texttt{newline} and
    \texttt{display} primitives shown in Exercise 1.22.  Then find a solution
    to $x^x=1000$ by finding a fixed point of $x\rightarrow\log{1000}/\log{x}$.
    (Use Scheme’s primitive \texttt{log} procedure, which computes natural
    logarithms.) Compare the number of steps this takes with and without
    average damping. (Note that you cannot start fixed-point with a guess of 1,
    as this would cause division by $\log{1}=0$.)
\end{quote}

\lstinputlisting[firstline=6,lastline=23]{ch1/ex1.36.scm}
Without average damping, it takes 34 guesses to get to 4 decimal places of
accuracy.

\lstinputlisting[firstline=32,lastline=35]{ch1/ex1.36.scm}
With average damping, it takes 11 guesses.

\section{Exercise 1.37}
\begin{quote}
    \begin{enumerate}
        \item An infinite continued fraction is an expression of the form
            \begin{equation*}
                f=\frac{N_1}{D_1+\frac{N_2}{D_2+\frac{N_3}{D_3+\ldots}}}.
            \end{equation*}
            As an example, one can show that the infinite continued fraction
            expansion with the $N_i$ and the $D_i$ all equal to 1 produces
            $1/\phi$, where $\phi$ is the golden ratio (described in 1.2.2).
            One way to approximate an infinite continued fraction is to
            truncate the expansion after a given number of terms. Such a
            truncation -- a so-called finite continued fraction
            \emph{k-term finite continued fraction} -- has the form
            \begin{equation*}
                f=\frac{N_1}{D_1+\frac{N_2}{\ddots+\frac{N_k}{D_k}}}.
            \end{equation*}
            Suppose that \texttt{n} and \texttt{d} are procedures of one
            argument (the term index $i$) that return the $N_i$ and $D_i$ of
            the terms of the continued fraction. Define a procedure
            \texttt{cont-frac} such that evaluating \texttt{(cont-frac n d k)}
            computes the value of the $k$-term finite continued fraction. Check
            your procedure by approximating $1/\phi$ using
            \begin{lstlisting}
(cont-frac (lambda (i) 1.0)
           (lambda (i) 1.0)
           k)
            \end{lstlisting}
            for successive values of \texttt{k}. How large must you make
            \texttt{k} in order to get an approximation that is accurate to 4
            decimal places?
        \item If your \texttt{cont-frac} procedure generates a recursive
            process, write one that generates an iterative process. If it
            generates an iterative process, write one that generates a
            recursive process.
    \end{enumerate}
\end{quote}

\subsection{Iterative}
\lstinputlisting[firstline=4,lastline=17]{ch1/ex1.37.scm}

\subsection{Recursive}
\lstinputlisting[firstline=23,lastline=31]{ch1/ex1.37.scm}

\section{Exercise 1.38}
\begin{quote}
    In 1737, the Swiss mathematician Leonhard Euler published a memoir
    \textit{De Fractionibus Continuis}, which included a continued fraction
    expansion for $e−2$, where $e$ is the base of the natural logarithms. In
    this fraction, the $N_i$ are all 1, and the $D_i$ are successively 1, 2, 1,
    1, 4, 1, 1, 6, 1, 1, 8, \ldots. Write a program that uses your
    \texttt{cont-frac} procedure from Exercise 1.37 to approximate $e$, based
    on Euler’s expansion.
\end{quote}

\lstinputlisting[firstline=13,lastline=20]{ch1/ex1.38.scm}

\section{Exercise 1.39}
\begin{quote}
    A continued fraction representation of the tangent function was published
    in 1770 by the German mathematician J.H. Lambert:
    \begin{equation*}
        \tan{x}= \frac{x}{1-\frac{x^2}{3-\frac{x^2}{5-\ldots}}},
    \end{equation*}
    where $x$ is in radians. Define a procedure \texttt{(tan-cf x k)} that
    computes an approximation to the tangent function based on Lambert’s
    formula. $k$ specifies the number of terms to compute, as in Exercise 1.37.
\end{quote}

\lstinputlisting[firstline=13,lastline=19]{ch1/ex1.39.scm}

\section{Exercise 1.40}
\begin{quote}
    Define a procedure \texttt{cubic} that can be used together with the
    \texttt{newtons-method} procedure in expressions of the form
    \begin{lstlisting}
(newtons-method (cubic a b c) 1)
    \end{lstlisting}
    to approximate zeros of the cubic $x^3+ax^2+bx+c$.
\end{quote}

\lstinputlisting[firstline=42,lastline=45]{ch1/ex1.40.scm}

\section{Exercise 1.41}
\begin{quote}
    Define a procedure \texttt{double} that takes a procedure of one argument
    as argument and returns a procedure that applies the original procedure
    twice. For example, if \texttt{inc} is a procedure that adds 1 to its
    argument, then \texttt{(double inc)} should be a procedure that adds 2.
    What value is returned by
    \begin{lstlisting}
(((double (double double)) inc) 5)
    \end{lstlisting}
\end{quote}

\lstinputlisting[firstline=3, lastline=4]{ch1/ex1.41.scm}

The value of \texttt{(((double (double double)) inc) 5)} is 21. It results in
a procedure that applies \texttt{inc} 16 times being applied to 5.

\section{Exercise 1.42}
\begin{quote}
    Let $f$ and $g$ be two one-argument functions. The composition $f$ after
    $g$ is defined to be the function $x\rightarrow f(g(x))$. Define a
    procedure \texttt{compose} that implements composition. For example, if
    \texttt{inc} is a procedure that adds 1 to its argument,
    \begin{lstlisting}
((compose square inc) 6)
49
    \end{lstlisting}
\end{quote}

\lstinputlisting[firstline=3,lastline=4]{ch1/ex1.42.scm}

\section{Exercise 1.43}
\begin{quote}
    If $f$ is a numerical function and $n$ is a positive integer, then we can
    form the $n^{\textrm{th}}$ repeated application of $f$, which is defined to
    be the function whose value at $x$ is $f(f(\ldots(f(x))\ldots))$. For
    example, if $f$ is the function $x\rightarrow x+1$, then the
    $n^{\textrm{th}}$ repeated application of $f$ is the function $x\rightarrow
    x+n$. If $f$ is the operation of squaring a number, then the
    $n^{\textrm{th}}$ repeated application of $f$ is the function that raises
    its argument to the $2^n$-th power. Write a procedure that takes as inputs
    a procedure that computes $f$ and a positive integer $n$ and returns the
    procedure that computes the $n^{\textrm{th}}$ repeated application of $f$.
    Your procedure should be able to be used as follows:
    \begin{lstlisting}
((repeated square 2) 5)
625
    \end{lstlisting}
\end{quote}

\lstinputlisting[firstline=6,lastline=15]{ch1/ex1.43.scm}

\section{Exercise 1.44}
\begin{quote}
    The idea of \emph{smoothing} a function is an important concept in signal
    processing. If $f$ is a function and $dx$ is some small number, then the
    smoothed version of $f$ is the function whose value at a point $x$ is the
    average of $f(x-dx)$, $f(x)$, and $f(x+dx)$. Write a procedure
    \texttt{smooth} that takes as input a procedure that computes $f$ and
    returns a procedure that computes the smoothed $f$. It is sometimes valuable
    to repeatedly smooth a function (that is, smooth the smoothed function, and
    so on) to obtain the \emph{n-fold smoothed function}. Show how to generate
    the n-fold smoothed function of any given function using \texttt{smooth} and
    \texttt{repeated} from Exercise 1.43.
\end{quote}

\lstinputlisting[firstline=15,lastline=23]{ch1/ex1.44.scm}

\section{Exercise 1.45}
\begin{quote}
    We saw in 1.3.3 that attempting to compute square roots by naively finding
    a fixed point of $y\rightarrow x/y$ does not converge, and that this can be
    fixed by average damping. The same method works for finding cube roots as
    fixed points of the average-damped $y\rightarrow x/y^2$. Unfortunately, the
    process does not work for fourth roots -- a single average damp is not
    enough to make a fixed-point search for $y\rightarrow x/y^3$ converge. On
    the other hand, if we average damp twice (i.e., use the average damp of the
    average damp of $y\rightarrow x/y^3$) the fixed-point search does converge.
    Do some experiments to determine how many average damps are required to
    compute $n^{\textrm{th}}$ roots as a fixed-point search based upon repeated
    average damping of $y\rightarrow x/y^{n-1}$.  Use this to implement a
    simple procedure for computing $n^{\textrm{th}}$ roots using
    \texttt{fixed-point}, \texttt{average-damp}, and the \texttt{repeated}
    procedure of Exercise 1.43.  Assume that any arithmetic operations you need
    are available as primitives.
\end{quote}

\lstinputlisting[firstline=41,lastline=52]{ch1/ex1.45.scm}

\section{Exercise 1.46}
\begin{quote}
    Several of the numerical methods described in this chapter are instances of
    an extremely general computational strategy known as \emph{iterative
    improvement}.  Iterative improvement says that, to compute something, we
    start with an initial guess for the answer, test if the guess is good
    enough, and otherwise improve the guess and continue the process using the
    improved guess as the new guess. Write a procedure
    \texttt{iterative-improve} that takes two procedures as arguments: a method
    for telling whether a guess is good enough and a method for improving a
    guess. \texttt{Iterative-improve} should return as its value a procedure
    that takes a guess as argument and keeps improving the guess until it is
    good enough. Rewrite the \texttt{sqrt} procedure of 1.1.7 and the
    \texttt{fixed-point} procedure of 1.3.3 in terms of
    \texttt{iterative-improve}.
\end{quote}

\lstinputlisting[firstline=11,lastline=39]{ch1/ex1.46.scm}

It would have been a bit more elegant to define \texttt{iterative-improve} in
terms of a named lambda using the \texttt{rec} keyword, but that hasn't been
covered yet. I also would have preferred to just using \texttt{close-enough?}
for the \texttt{good-enough?} argument and define \texttt{iterative-improve} to
pass 2 arguments to \texttt{good-enough?}. However, the problem specifies a
one-argument interface.

\end{document}
