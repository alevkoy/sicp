\documentclass{article}
\author{Abraham Levkoy}
\title{SICP exercises, section 1.2}

\usepackage{mystyle}

\begin{document}
\maketitle

\section{Exercise 1.29}
\begin{quote}
    Simpson's Rule is a more accurate method of numerical integration than the
    method illustrated above. Using Simpson's Rule, the integral of a function
    $f$ between $a$ and $b$ is approximated as
    \begin{equation*}
        \frac{h}{3}(y_0+4y_1+2y_2+4y_3+2y_4+\cdots+2y_{n-2}+4y_{n-1}+y_n),
    \end{equation*}
    where $h=(b−a)/n$, for some even integer $n$, and $y_k=f(a+kh)$.
    (Increasing $n$ increases the accuracy of the approximation.) Define a
    procedure that takes as arguments $f$, $a$, $b$, and $n$ and returns the
    value of the integral, computed using Simpson’s Rule. Use your procedure to
    integrate \texttt{cube} between 0 and 1 (with $n=100$ and $n=1000$), and
    compare the results to those of the \texttt{integral} procedure shown
    above.
\end{quote}

\lstinputlisting[firstline=4,lastline=26]{ch1/ex1.29.scm}

\begin{lstlisting}
> (integral cube 0 1 .01)
.24998750000000042
> (simpson cube 0 1 100)
.25
> (integral cube 0 1 .001)
.2467166666666667
> (simpson cube 0 1 1000)
.25
\end{lstlisting}

\texttt{simpson} finds the exact value of the integral ($\frac{1}{4}$) and is
thus more accurate than \texttt{integral}. In fact, it is spookily accurate. It
seems to produce results using only 2 intervals that are as accurate as
\texttt{integral} with 100 intervals.

\end{document}
