\documentclass{article}
\author{Abraham Levkoy}
\title{SICP exercises, section 1.1}

\usepackage{color}
\usepackage{listings}

\begin{document}
\maketitle

\lstset{
    language=Lisp,
    otherkeywords={define,if,else},
    backgroundcolor=\color{white},
    basicstyle=\ttfamily,
    keywordstyle=\color{blue}\ttfamily,
    stringstyle=\color{red}\ttfamily,
    commentstyle=\color{magenta}\\ttfamily
}

\section{Exercise 1.1}
\begin{quote}
    Below is a sequence of expressions. What is the result printed by the
    interpreter in response to each expression? Assume that the sequence is to
    be evaluated in the order in which it is presented.
\end{quote}

\noindent \lstinline|10| \\
\emph{10}

\noindent \lstinline|(+ 5 3 4)| \\
\emph{12}

\noindent \lstinline|(- 9 1)| \\
\emph{8}

\noindent \lstinline|(/ 6 2)| \\
\emph{3}

\noindent \lstinline|(+ (* 2 4) (- 4 6))| \\
\emph{6}

\noindent \lstinline|(define a 3)| \\
\emph{[stored value a]}

\noindent \lstinline|(define b (+ a 1))| \\
\emph{[stored value b]}

\noindent \lstinline|(+ a b (* a b))| \\
\emph{19}

\noindent \lstinline|(= a b)| \\
\emph{\#f}

\begin{lstlisting}
(if (and (> b a) (< b (* a b)))
    b
    a)
\end{lstlisting}
\emph{4}

\begin{lstlisting}
(cond ((= a 4) 6)
      ((= b 4) (+ 6 7 a))
      (else 25))
\end{lstlisting}
\emph{16}

\noindent \lstinline|(+ 2 (if (> b a) b a))| \\
\emph{6}

\begin{lstlisting}
(* (cond ((> a b) a)
         ((< a b) b)
         (else -1))
   (+ a 1))
\end{lstlisting}
\emph{16}

\section{Exercise 1.2}
\begin{quote}
    Translate the following expression into prefix form:
    \begin{math}
        \frac{5+4+(2-(3-6+\frac{4}{5}))}{3(6-2)(2-7)}
    \end{math}.
\end{quote}

\lstinline|(/ (+ 5 4 (- 2 (- 3 (+ 6 (/ 4 5))))) (* 3 (- 6 2) (- 2 7)))|

\section{Exercise 1.3}
\begin{quote}
	Define a procedure that takes three numbers as arguments and returns the sum
	of the squares of the two larger numbers.
\end{quote}

\lstinputlisting[firstline=2]{ch1/ex1.3.scm}

\section{Exercise 1.4}
\begin{quote}
	Observe that our model of evaluation allows for combinations whose operators
    are compound expressions. Use this observation to describe the behavior of
    the following procedure:
    \begin{lstlisting}
(define (a-plus-abs-b a b)
  ((if (> b 0) + -) a b))
    \end{lstlisting}
\end{quote}

If $b$ is positive, the procedure returns \verb|(+ a b)|; otherwise, it returns
\verb|(- a b)|. Thus, it returns $a + |b|$.

\section{Exercise 1.5}
\begin{quote}
    Ben Bitdiddle has invented a test to determine whether the interpreter he is
    faced with is using applicative-order evaluation or normal-order evaluation.
    He defines the following two procedures:
    \begin{lstlisting}
(define (p) (p))

(define (test x y) 
  (if (= x 0) 
      0 
      y))
	\end{lstlisting}
	Then he evaluates the expression \verb|(test 0 (p))|. What behavior
	will Ben observe with an interpreter that uses applicative-order evaluation?
	What behavior will he observe with an interpreter that uses normal-order
	evaluation? Explain your answer. (Assume that the evaluation rule for the
	special form \verb|if| is the same whether the interpreter is using
	normal or applicative order: The predicate expression is evaluated first,
	and the result determines whether to evaluate the consequent or the
	alternative expression.)
\end{quote}

Attempting to evaluate \verb|(p)| will recurse infinitely and thus never
complete. An interpreter using normal-order evaluation will expand the test
operator before evaluating the operands. Then, because $x$ is 0, it will
evaluate the consequent, returning 0 as a result. Conversely, an interpreter
using applicative order will attempt to evaluation the operands of test before
substituting them for test's formal parameters. Consquently, it will attempt to
evaluate \verb|(p)| and run forever without result.

\section{Exercise 1.6}
\begin{quote}
    Alyssa P. Hacker doesn’t see why \verb|if| needs to be provided as a
    special form.  “Why can’t I just define it as an ordinary procedure in
    terms of \verb|cond|?” she asks. Alyssa’s friend Eva Lu Ator claims
    this can indeed be done, and she defines a new version of \verb|if|:
	\begin{lstlisting}
(define (new-if predicate 
                then-clause 
                else-clause)
  (cond (predicate then-clause)
        (else else-clause)))
	\end{lstlisting}
	Eva demonstrates the program for Alyssa:
	\begin{lstlisting}
(new-if (= 2 3) 0 5)
5

(new-if (= 1 1) 0 5)
0
	\end{lstlisting}
    Delighted, Alyssa uses \verb|new-if| to rewrite the square-root
    program:
	\begin{lstlisting}
(define (sqrt-iter guess x)
  (new-if (good-enough? guess x)
          guess
          (sqrt-iter (improve guess x) x)))
	\end{lstlisting}
	What happens when Alyssa attempts to use this to compute square roots?
	Explain.
\end{quote}

Using \verb|new-if| to compute square roots will result in infinite
recursion. The square root procedure refines the guess recursively until it is
good enough, at which point it stops recursiing and returns. It achieves this
by exploiting the property of the \verb|if| special form to evaluate only
one of the the consequent or the alternative based on the predicate.
\verb|new-if| is a procedure, so all of its arguments are evaluated before
it is. This means that the recursive alternative case will always be evaluated,
even when the predicate to return the nonrecursive consequent case has already
been satisfied. Thus, using \verb|sqrt-iter| with \verb|new-if| will
result in infinite recursion.

\section{Exercise 1.7}
\begin{quote}
    The \verb|good-enough?| test used in computing square roots will not be
    very effective for finding the square roots of very small numbers. Also, in
    real computers, arithmetic operations are almost always performed with
    limited precision. This makes our test inadequate for very large numbers.
    Explain these statements, with examples showing how the test fails for
    small and large numbers. An alternative strategy for implementing
    \verb|good-enough?| is to watch how guess changes from one iteration to the
    next and to stop when the change is a very small fraction of the guess.
    Design a square-root procedure that uses this kind of end test. Does this
    work better for small and large numbers?
\end{quote}

\verb|good-enough?| returns true when the square of the guess is within 0.001
of the argument \verb|x|. This is inadequate for very small numbers, because
the squares of the guesses will get below .001 (and consequently within .001 of
\verb|x|) before they get reasonably close to the true square root of \verb|x|.
For instance, \verb|(sqrt .001)| evaluates to approximately .04125, which
squares to about .0017, almost twice .001. Inputs an order of magnitude lower
than .001 or less produce results that square to approximately .001.

For very large numbers, \verb|good-enough?| will never return true, so the
algorithm will iterate forever. This is because the arithmetic operations
involved in improving the guess can only produce results of a certain
precision. When two very large numbers are as close to each other without being
equal as the representation allows, their squares may differ by more than .001,
meaning that the square root of a number between those two squares would not be
representable by the interpreter to the satisfaction of \verb|good-enough?|. For
instance, \verb|(sqrt 1234567890123456789012345678901234567890)| runs forever.

The alternative test can be implemented as
\lstinputlisting[firstline=4,lastline=5]{ch1/ex1.7.scm}

This test causes \verb|sqrt| to stop recursing when the guess stops changing
significantly, resulting in reasonable results for very large and very small
inputs.

\section{Exercise 1.8}
\begin{quote}
    Newton’s method for cube roots is based on the fact that if y is an
    approximation to the cube root of x , then a better approximation is given
    by the value $$\frac{\frac{x}{y^2} + 2y}{3}.$$  Use this formula to
    implement a cube-root procedure analogous to the square-root procedure.
\end{quote}

\lstinputlisting[firstline=2]{ch1/ex1.8.scm}

\section{Exercise 1.9}
\begin{quote}
    Each of the following two procedures defines a method for adding two
    positive integers in terms of the procedures \verb|inc|, which increments
    its argument by 1, and \verb|dec|, which decrements its argument by 1.
    \begin{lstlisting}
(define (+ a b)
  (if (= a 0)
      b
      (inc (+ (dec a) b))))

(define (+ a b)
  (if (= a 0)
      b
      (+ (dec a) (inc b))))
    \end{lstlisting}
    Using the substitution model, illustrate the process generated by each
    procedure in evaluating \verb|(+ 4 5)|. Are these processes iterative or
    recursive?
\end{quote}

The first procedure generates a recursive process:
\begin{verbatim}
(+ 4 5)
(inc (+ 3 5))
(inc (inc (+ 2 5)))
(inc (inc (inc (+ 1 5))))
(inc (inc (inc (inc (+ 0 5)))))
(inc (inc (inc (inc 5))))
(inc (inc (inc 6)))
(inc (inc 7))
(inc 8)
9
\end{verbatim}

The second procedure generates an iterative process:
\begin{verbatim}
(+ 4 5)
(+ 3 6)
(+ 2 7)
(+ 1 8)
(+ 0 9)
9
\end{verbatim}

\section{Exercise 1.10}
\begin{quote}
	The following procedure computes a mathematical function called Ackermann’s
	function.
    \begin{lstlisting}
(define (A x y)
  (cond ((= y 0) 0)
        ((= x 0) (* 2 y))
        ((= y 1) 2)
        (else (A (- x 1)
                 (A x (- y 1))))))
    \end{lstlisting}
    What are the values of the following expressions?

    \verb|(A 1 10)|\\
    \verb|(A 2 4)|\\
    \verb|(A 3 3)|\\
\end{quote}

\begin{minipage}{\linewidth}
\begin{verbatim}
(A 1 10)
(A 0 (A 1 9))
(A 0 (A 0 (A 1 8)))
(A 0 (A 0 (A 0 (A 1 7))))
(A 0 (A 0 (A 0 (A 0 (A 1 6)))))
(A 0 (A 0 (A 0 (A 0 (A 0 (A 1 5))))))
(A 0 (A 0 (A 0 (A 0 (A 0 (A 0 (A 1 4)))))))
(A 0 (A 0 (A 0 (A 0 (A 0 (A 0 (A 0 (A 1 3))))))))
(A 0 (A 0 (A 0 (A 0 (A 0 (A 0 (A 0 (A 0 (A 1 2)))))))))
(A 0 (A 0 (A 0 (A 0 (A 0 (A 0 (A 0 (A 0 (A 0 (A 1 1))))))))))
(A 0 (A 0 (A 0 (A 0 (A 0 (A 0 (A 0 (A 0 (A 0 2)))))))))
(A 0 (A 0 (A 0 (A 0 (A 0 (A 0 (A 0 (A 0 4))))))))
(A 0 (A 0 (A 0 (A 0 (A 0 (A 0 (A 0 8)))))))
(A 0 (A 0 (A 0 (A 0 (A 0 (A 0 16))))))
(A 0 (A 0 (A 0 (A 0 (A 0 32)))))
(A 0 (A 0 (A 0 (A 0 64))))
(A 0 (A 0 (A 0 128)))
(A 0 (A 0 256))
(A 0 512)
1024
\end{verbatim}\par
\end{minipage}
% minipage does not seem to have a blank line after it like a normal paragraph
\vspace{\baselineskip}

\begin{minipage}{\linewidth}
\begin{verbatim}
(A 2 4)
(A 1 (A 2 3))
(A 1 (A 1 (A 2 2)))
(A 1 (A 1 (A 1 (A 2 1))))
(A 1 (A 1 (A 1 2)))
(A 1 (A 1 (A 0 (A 1 1))))
(A 1 (A 1 (A 0 2)))
(A 1 (A 1 4))
...
(A 1 16)
...
65536
\end{verbatim}
\end{minipage}
\vspace{\baselineskip}

\begin{minipage}{\linewidth}
\begin{verbatim}
(A 3 3)
(A 2 (A 3 2))
(A 2 (A 2 (A 3 1)))
(A 2 (A 2 2))
...
(A 2 4)
...
65536
\end{verbatim}\par
\end{minipage}
\vspace{\baselineskip}

\begin{quote}
    Consider the following procedures, where A is the procedure defined above:

    \begin{lstlisting}
(define (f n) (A 0 n))
(define (g n) (A 1 n))
(define (h n) (A 2 n))
(define (k n) (* 5 n n))
    \end{lstlisting}

    Give concise mathematical definitions for the functions computed by the
    procedures \verb|f|, \verb|g|, and \verb|h| for positive integer values of
    $n $. For example, \verb|(k n)| computes $5n^2$.
\end{quote}

\verb|(f n)| calculates $2n$.

\verb|(g n)| calculates $2^n$.

\verb|(h n)| calculates $2^{2^2}\ldots$ ($n$ levels of exponentiation).

\end{document}
